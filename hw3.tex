%%%%%%%%%%%%%%%%%%%%%%%%%%%%%%%%%%%%%%%%%%%%%%%%%%%%%%%%%%%%%%%%%%%%%%%%%%%%%%%%%%%%
%%%%%%%%%%%%%%%%%%%free online editor available at%%%%%%%%%%%%%%%%%%%%%%
%%%%%https://www.writelatex.com/%%%%%%%%%%%%%%%%%%%%%%%%%%%%%%%%%%%%%%%%%%

\documentclass[10pt,leqno ]{article}
    %The document class defines the master templates, the structure of the document, 
    %and lays out the types of 
    %objects that can be manipulated for this type of document. 
    %the brackets contain basic options that will be applied globally (throughout
    %the document). Here, we specify a 10pt font, and when we number an equation, the 
    %number will be on the left.
    %The document class is a file ***.cls. You will probably never have to edit or create
    % a .cls file. There are many available on the internet for your use.

%%%%%%%%%%%%%%%%%%%%%%%%%%%%%%%%%%%%%%%%%%%%%%%%%%%%%%%%%%%%%%%%%%%%%%%%%%%%%%%%%%%%%%%%%%%%%%%%%%%%%%%%%%%%%%%%%%%%%%%%%%%%%%%%%%%%%%%%%%%%%%%%%%%%%%%%%%%%%%%%%%%%%%%%%%%%%%%%%%%%%%%%%%%%%%%%%%%%%%%%%%%%%%%%%%%%%%%%%%%%%%%%%%%%%%%%%%%%%%%%%%%%%%%%%%%%
\usepackage{amsfonts}
\usepackage[shortlabels]{enumitem}
\usepackage{amssymb}
\usepackage{amsmath}
\usepackage{times}
\usepackage{amsthm}
\usepackage{hyperref}
%\usepackage{homework}
    %packages control the ``style'' or look of the document. These come in the form of 
    %files ***.sty. The package ``homework'' above was created by me. The other packages
    %are very common for this type of document. You can google to learn more about what
    %they can do, and what options they give you. For example
\usepackage{textcomp}
\usepackage[margin=1.5in]{geometry}
    %the geometry package lets you customize the margins of your document.
    % and the 

\usepackage{forest}
  % For drawing the tree illustrating the sample space  
\usepackage{setspace}
    %package gives us the ability to set the line spacing.
\usepackage{moreenum}   % for greek letters in enum
\usepackage{xcolor}

\newtheorem{theorem}{Theorem}
\theoremstyle{definition} 
\newtheorem{problem}[theorem]{Problem}
    %these set up environments for listing things. The numbering is automatic.
\usepackage{float}
\restylefloat{table}

\newenvironment{solution}[1][Solution]{\begin{doublespace}\textbf{#1.}\quad }{\ \rule{0.5em}{0.5em}\end{doublespace}}
    %this is the environment for writing solutions. Doble spaced, with an end of proof
    %box at the end
    
\title{Discrete Math\\
CSCI 150, Fall 2020\\
Homework 1}
\author{Guy Matz \\
Hunter College}
    %above is the information that goes in the title. Notice the { and }. 
    %the double slashes \\ mean start a new line.


\begin{document} %this means end the preamble (stuff controling the styles above and
%start the content of the document. We can make adjustments as we go. For example,
%\maketitle
\begin{problem} Find $(\{2,4,6\} \cup  \{6,4\}) \cap \{4,6,8\}$
\\
\Large
\begin{align*}
(\{2,4,6\} \cup  \{6,4\}) \cap \{4,6,8\} &= \{2,4,6\} \cap \{4,6,8\}\\
        &=  \{4,6\}
\end{align*}
\end{problem}
\newpage

\begin{problem} Find $\{1,2,3\} \times \{0,1\}$
\\
\Large
$$\left\{ (1,0), (1,1), (2,0), (2,1), (3,0), (3,1) \right\}$$
\end{problem}
\newpage
\begin{problem} Find $P(\{7,8,9\})-P({7,9}$), where $P(S)$ is the power set of $S$ (the set of all subsets of $S$), and $S-T = \{x|x \in S \text{ and } x \notin T\}$.
\\\\
\Large
$$\mathcal{P}(\{7,8,9\}) - \mathcal{P}(\{7,9\}) = \left\{ \{8\}, \{7,8\}, \{8, 9\} , \{7,8,9\} \right\}$$
\end{problem}
\newpage
\begin{problem} Find $\mathcal{P}(\mathcal{P}(\{2\}))$
\\
\Large
$$\mathcal{P}(\{2\}) = \{\emptyset, \{2\}\}$$
$$\mathcal{P}(\{\emptyset, \{2\}\}) = \{\emptyset, \{ \emptyset \}, \{\{2\}\}, \{\emptyset, \{2\}\}\}$$
\end{problem}
\newpage
\begin{problem} Describe the set of all 3-digit positive integers using set notation.
\\
\Large
$$\{100, 101, 102, ... , 999\}$$
\center{or}
$$ \{ x \in \mathbb{N}| x \geq 100 \text{ and } x \leq 999 \}$$
\end{problem}
\newpage
\begin{problem} Describe the set of all numbers divisible by 3 or 5 using set notation.
\\
\Large
$$\{x | x \in \mathbb{Z} \wedge (3 | x \wedge 5|x) \}$$
\end{problem}
\newpage

\begin{problem} Which  of  the  following  is  true?   (there  is  only  one):
\\\\
\begin{itemize}
\item $\emptyset=\{\emptyset\}$
\item $|\emptyset|=  0 \color{red}\checkmark$
\item $|P(\emptyset)|= 0$
\item $\emptyset \in \{ \}$
\end{itemize}
\Large

\end{problem}
\newpage

\begin{problem} Let $P$ be a set of people, $U$ a set of umbrellas, and $D$ the set of days of the year.  A day can be either good or bad.  Define $f:P \to U$ such that $f(p) =u$ means person $p$ owns umbrella $u$ (each person owns exactly one umbrella).  Write the following statement using symbols in $\{\forall,\exists,\implies\}$
\begin{center}
“If two people own the same umbrella, then all days are bad.”
\end{center}
Hint: Express “two people own the same umbrella” and “all days are bad”separately, then combine them.
\Large
$$ \exists p_1, p_2 \in P,  p_1 \neq p_2 \text{ and } f(p_1) = f(p_2)  \implies \forall d \in D, d \text{ is bad }$$
\end{problem}
\newpage

\begin{problem} We have seen in class and recitation, that the power set of a set $S$ has cardinality $2^{|S|}$ (assuming $S$ is finite).  Let $A$ and $B$ be two finite sets.  Is it possible that $|\mathcal{P}(A \times B)|=|\mathcal{P}(A)|\cdot|\mathcal{P}(B)|$?  If Yes, give an example, if No say why.
\\\\
\Large
This is not possible.  Consider two sets $A$ and $B$, where $|A| = n$, and $|B| = m$.  Then $|\mathcal{P}(A \times B)| = 2^{nm}$ and $|\mathcal{P}(A)| \cdot |\mathcal{P}(B)| = 2^{n+m}$.  However, $nm = n+m$ has no solution, therefore it is not possible that $|\mathcal{P}(A \times B)|=|\mathcal{P}(A)|\cdot|\mathcal{P}(B)|$.
\end{problem}
\newpage

\begin{problem} You are on a trip and you find 3 kinds of postcards $K=\{A, B, C\}$.  You have 12 friends.  You want to send postcards to your friends while observing the following rules:
\begin{itemize}
\item Each friend must receive at least one postcard\\
\item No friend can receive two or more postcards of the same kind
\end{itemize}
In how many ways can you send the postcards?  Hint:  Think about the powerset of $K$ to figure out the possibilities for each friend, then use the product rule.
\Large
\\\\
For each friend, there are $2^n-1$ ways we can pick from $n$ postcards without picking none.  So for 12 people and 3 postcards we have $(2^n-1)^{12} = 7^{12} = 1,384,287,201$.
\end{problem}
\newpage

\begin{problem} Consider the following 4 functions.
\begin{itemize}
\item $f:\mathbb{N} \to \mathbb{R} ,f(x) = 1/x$\\
\item $g:\mathbb{N} \times \mathbb{N} \to \mathbb{Q}+,g(x, y) =x/y,$ where $\mathbb{Q}+$ is the set of all rational numbers greater than zero (the positive rational numbers)\\
\item $h:\mathbb{Z} \to \mathbb{Z} ,h(x) = x^2$\\
\item $w:\mathbb{R} \to \mathbb{R}, w(x) = 3x+ 1$.
\end{itemize}
Place them appropriately in the following four categories:  one-to-one and/or onto
\\
\Large

\begin{table}[H]
\begin{tabular}{|c|c|c|}
\hline
Function & 1-1 ? & onto ? \\ \hline
$f$ & Y & n \\ \hline
$g$ & n & Y \\ \hline
$h$ & n &  n\\ \hline
$w$ & Y &  Y\\ \hline
\end{tabular}
\end{table}

\end{problem}
\end{document}
