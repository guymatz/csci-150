%%%%%%%%%%%%%%%%%%%%%%%%%%%%%%%%%%%%%%%%%%%%%%%%%%%%%%%%%%%%%%%%%%%%%%%%%%%%%%%%%%%%
%%%%%%%%%%%%%%%%%%%free online editor available at%%%%%%%%%%%%%%%%%%%%%%
%%%%%https://www.writelatex.com/%%%%%%%%%%%%%%%%%%%%%%%%%%%%%%%%%%%%%%%%%%

\documentclass[10pt,leqno ]{article}
    %The document class defines the master templates, the structure of the document, 
    %and lays out the types of 
    %objects that can be manipulated for this type of document. 
    %the brackets contain basic options that will be applied globally (throughout
    %the document). Here, we specify a 10pt font, and when we number an equation, the 
    %number will be on the left.
    %The document class is a file ***.cls. You will probably never have to edit or create
    % a .cls file. There are many available on the internet for your use.

%%%%%%%%%%%%%%%%%%%%%%%%%%%%%%%%%%%%%%%%%%%%%%%%%%%%%%%%%%%%%%%%%%%%%%%%%%%%%%%%%%%%%%%%%%%%%%%%%%%%%%%%%%%%%%%%%%%%%%%%%%%%%%%%%%%%%%%%%%%%%%%%%%%%%%%%%%%%%%%%%%%%%%%%%%%%%%%%%%%%%%%%%%%%%%%%%%%%%%%%%%%%%%%%%%%%%%%%%%%%%%%%%%%%%%%%%%%%%%%%%%%%%%%%%%%%
\usepackage{amsfonts}
\usepackage[shortlabels]{enumitem}
\usepackage{amssymb}
\usepackage{amsmath}
\usepackage{times}
\usepackage{amsthm}
\usepackage{hyperref}
%\usepackage{homework}
    %packages control the ``style'' or look of the document. These come in the form of 
    %files ***.sty. The package ``homework'' above was created by me. The other packages
    %are very common for this type of document. You can google to learn more about what
    %they can do, and what options they give you. For example
\usepackage{textcomp}
\usepackage[margin=1.5in]{geometry}
    %the geometry package lets you customize the margins of your document.
    % and the 

\usepackage{forest}
  % For drawing the tree illustrating the sample space  
\usepackage{setspace}
    %package gives us the ability to set the line spacing.
\usepackage{moreenum}   % for greek letters in enum
\usepackage{xcolor}

\newtheorem{theorem}{Theorem}
\theoremstyle{definition} 
\newtheorem{problem}[theorem]{Problem}
    %these set up environments for listing things. The numbering is automatic.
\usepackage{float}
\restylefloat{table}
\usepackage{mathtools}
\DeclarePairedDelimiter{\ceil}{\lceil}{\rceil}
\DeclarePairedDelimiter{\floor}{\lfloor}{\rfloor}

\newenvironment{solution}[1][Solution]{\begin{doublespace}\textbf{#1.}\quad }{\ \rule{0.5em}{0.5em}\end{doublespace}}
    %this is the environment for writing solutions. Doble spaced, with an end of proof
    %box at the end
    
\title{Discrete Math\\
CSCI 150, Fall 2020\\
Homework 1}
\author{Guy Matz \\
Hunter College}
    %above is the information that goes in the title. Notice the { and }. 
    %the double slashes \\ mean start a new line.


\begin{document} %this means end the preamble (stuff controling the styles above and
%start the content of the document. We can make adjustments as we go. For example,
%\maketitle
\begin{problem} Factor 10! into primes.
\\\\
\Large
\begin{align*}
10! &= 10 \cdot 9 \cdot 8 \cdot 7  \cdot  6  \cdot  5  \cdot 4 \cdot 3 \cdot 2 \cdot 1\\
    &= 5 \cdot 2 \cdot 3 \cdot 3 \cdot 2 \cdot 2 \cdot 2 \cdot 7 \cdot 3 \cdot 2 \cdot 5 \cdot 2 \cdot 2 \cdot 3 \cdot 2\\
    &= 7 \cdot 5^2 \cdot 3^4 \cdot 2^8
\end{align*}
\end{problem}
\newpage

\begin{problem} Find the greatest common divisor of 100 and 254 using prime factorization.
\\\\
\Large
Prime factors of 100: $2^2, 5^2$\\
Prime factors of 254: $2, 127$\\
\\
Hence $gcd(100, 254) = 2$


\end{problem}
\newpage

\begin{problem} Find  the  greatest  common  divisor  of  100  and  254  using  the  Euclidean algorithm.
\\\\
\Large
$$254 \; 100 \; 54 \; 46 \; 8 \; 6 \; \underline{2} \; 0$$
\end{problem}
\newpage

\begin{problem} Express the $gcd(100,254)$ as a linear combination of 100 and 254.
\\\\
\Large
$$254 * 13 - 100 * 33 = 2$$
\end{problem}
\newpage

\begin{problem} Show that if $a|bc$ and $gcd(a, b) = 1$ then $a|c$.
\\\\
\Large
Towards a contradiction, let's assume that if $a|bc$ and $gcd(a,b) = 1$ then $a \nmid c$.  

We have $n \cdot a = b \cdot c$, where $n \in \mathbb{N}$, 
then $n = \frac{b}{a} \cdot c$.  But then, if $c$ is not a multiple of $a$, 
we are left with a fraction, but $n$ must be an integer, so we have a contradiction.
\end{problem}
\newpage

\begin{problem} Let $F_n$ be the $n^{th}$ Fibonacci number.  What is the $gcd(F_{2021}, F_{2020})$?
\\\\
\Large
Since we know that $gcd(a+b, b) = gcd(a,b)$ we can say that \\
\begin{align*}
gcd(F_{2021}, F_{2020}) &= gcd(F_{2020} + F_{2019}, F_{2020}) \\
                        &= gcd(F_{2019}, F_{2020}) \\
                        &= gcd(F_{2019}, F_{2019} + F_{2018}) \\
                        &= gcd(F_{2019}, F_{2018}) \\
                        &= gcd(F_{2018} + F_{2017}, F_{2018}) \\
                        &= gcd(F_{2017}, F_{2018}) \\
                        & \vdots\\
                        &= gcd(F_2, F_1)\\
                        &= gdc(2, 1)\\
                        &= 1
\end{align*}
\end{problem}
\newpage

\begin{problem} Consider the number 60, it is factored into primes as $2^2 \cdot 3 \cdot 5$.  Observe now that $(2^0+ 2^1+ 2^2)(3^0+ 3^1)(5^0+ 5^1)$ gives the sum of all the divisors of 60 due to the distributive law.  Find the sum of all the divisors of 1000.
\\\\
\Large
1000 factored into primes is $2^3 \cdot 5^3$.  Then $(2^0 + 2^1 + 2^2 + 2^3)( 5^0 + 5^1 + 5^2 + 5^3) = 15 \cdot 156 = 2,340$
\end{problem}
\newpage

\begin{problem} Find the product of all the divisors of 1000.  This requires a little bit of thinking along the lines of the previous question.
\\\\
\Large
$$\prod_{i=0}^{3} \prod_{j=0}^{3}2^i \cdot 5^j$$
$$= 2^0 \cdot 5^0 \cdot 2^1 \cdot 5^0 \cdot 2^2 \cdot 5^0 \cdot 2^3 \cdot 5^0 \cdot 2^0 \cdot 5^1 \cdot 2^1 \cdot 5^1 \cdot 2^2 \cdot 5^1 \cdot 2^3 \cdot 5^1 \cdot$$
$$2^0 \cdot 5^2 \cdot 2^1 \cdot 5^2 \cdot 2^2 \cdot 5^2 \cdot 2^3 \cdot 5^2 \cdot 2^0 \cdot 5^3 \cdot 2^1 \cdot 5^3 \cdot 2^2 \cdot 5^3 \cdot 2^3 \cdot 5^3$$
$$ = 2^{24} \cdot 5^{24}$$
$$ = 1,000,000,000,000,000,000,000,000$$
\end{problem}
\newpage

Consider the following sequences starting at $a_0, a_1, \dots$:
$$5, -10, 20, -40, \dots$$
$$ 1, 7, 49, 343$$
\begin{problem} For each of the sequences above, find a recurrence of the form $a_n = Aa_{n-1}$ for $n \geq 1$, and solve for $a_n$ as a function of $n$.
\\\\
\begin{enumerate}[label=\alph*)]
\item $5, -10, 20, -40, \dots$
$$a_n = -2 \cdot a_{n-1}$$
We form the Auxiliary Equation: 
$$x^2 = -2x$$
which has two solutions 0 and -2.   Therefore, our solution for $a_n$ has the form:
$$c_1(0)^n + c_2(-2)^n$$
Let's determine $c_1$ and $c_2$ from one initial condition:
$$a_1 = c_1(0)^1 + c_2(-2)^1 = -2c_2 = -10$$
We obtain $c_2 = 5$.  Therefore,
$$a_n = 5(-2)^n$$
\\\\
\item $1, 7, 49, 343, \dots$
$$a_n = 7 \cdot a_{n-1}$$
We form the Auxiliary Equation: 
$$x^2 = 7x$$
which has two solutions 0 and 7.   Therefore, our solution for $a_n$ has the form:
$$c_1(0)^n + c_2(7)^n$$
Let's determine $c_1$ and $c_2$ from one initial condition:
$$a_1 = c_1(0)^1 + c_2(7)^1 = 7c_2 = 7$$
We obtain $c_2 = 1$.  Therefore,
$$a_n = 7^n$$
\end{enumerate}
\end{problem}
\newpage

\begin{problem} For each of the sequences above, find a recurrence of the form $a_n=Aa_{n-1} + Ba_{n-2}$ for $n \geq 2$, by considering the recurrence from part (a) for $a_n$ and $a_{n-1}$ ; the solution is not unique,  depending on how you combine recurrences,  so to make grading feasible, find the solution that corresponds to adding up the recurrences.
\\\\
\Large

\begin{enumerate}[label=\alph*)]
\item $5, -10, 20, -40, \dots$
$$a_n = -2 \cdot a_{n-1}$$
\begin{align*}
a_2 &= Aa_1 + Ba_0 = 20\\
    &= -10A   + 5B = 20\\
a_3 &= Aa_2 + Ba_1 = -40\\
    &= 20A  -10B   = -40\\
\end{align*}
\item $1, 7, 49, 343, \dots$
$$a_n = 7 \cdot a_{n-1}$$
\begin{align*}
a_2 &= Aa_1 + Ba_0 = 49\\
    &= 7A   + 1B = 49\\
a_3 &= Aa_2 + Ba_1 = 343\\
    &= 49A + 7B   = 343\\
\end{align*}
\end{enumerate}

\end{problem}
\newpage

\begin{problem} There are infinitely many recurrences of the form $a_n=Aa_{n-1}+Ba_{n-2}$ that work since we can write $a_n=cp^n+ 0 \cdot q^n$ for $q \neq p$.  Find a recurrence of the form $a_n=Aa_{n-1}+Ba_{n-2}$ for $n \geq 2$ that works for both sequences at the same time
\\\\
\Large
\begin{enumerate}[label=\alph*)]
\item $5, -10, 20, -40, \dots$
$$a_n = -2 \cdot a_{n-1}$$
\begin{align*}
a_2 &= Aa_1 + Ba_0 = 20\\
    &= -10A   + 5B = 20\\
a_3 &= Aa_2 + Ba_1 = -40\\
    &= 20A  -10B   = -40\\
\end{align*}
\item $1, 7, 49, 343, \dots$
$$a_n = 7 \cdot a_{n-1}$$
\begin{align*}
a_2 &= Aa_1 + Ba_0 = 49\\
    &= 7A   + 1B = 49\\
a_3 &= Aa_2 + Ba_1 = 343\\
    &= 49A + 7B   = 343\\
\end{align*}
\end{enumerate}
\end{problem}
\newpage
\end{document}
