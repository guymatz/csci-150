%%%%%%%%%%%%%%%%%%%%%%%%%%%%%%%%%%%%%%%%%%%%%%%%%%%%%%%%%%%%%%%%%%%%%%%%%%%%%%%%%%%%
%%%%%%%%%%%%%%%%%%%free online editor available at%%%%%%%%%%%%%%%%%%%%%%
%%%%%https://www.writelatex.com/%%%%%%%%%%%%%%%%%%%%%%%%%%%%%%%%%%%%%%%%%%

\documentclass[10pt,leqno ]{article}
    %The document class defines the master templates, the structure of the document, 
    %and lays out the types of 
    %objects that can be manipulated for this type of document. 
    %the brackets contain basic options that will be applied globally (throughout
    %the document). Here, we specify a 10pt font, and when we number an equation, the 
    %number will be on the left.
    %The document class is a file ***.cls. You will probably never have to edit or create
    % a .cls file. There are many available on the internet for your use.

%%%%%%%%%%%%%%%%%%%%%%%%%%%%%%%%%%%%%%%%%%%%%%%%%%%%%%%%%%%%%%%%%%%%%%%%%%%%%%%%%%%%%%%%%%%%%%%%%%%%%%%%%%%%%%%%%%%%%%%%%%%%%%%%%%%%%%%%%%%%%%%%%%%%%%%%%%%%%%%%%%%%%%%%%%%%%%%%%%%%%%%%%%%%%%%%%%%%%%%%%%%%%%%%%%%%%%%%%%%%%%%%%%%%%%%%%%%%%%%%%%%%%%%%%%%%
\usepackage{amsfonts}
\usepackage[shortlabels]{enumitem}
\usepackage{amssymb}
\usepackage{amsmath}
\usepackage{times}
\usepackage{amsthm}
\usepackage{hyperref}
%\usepackage{homework}
    %packages control the ``style'' or look of the document. These come in the form of 
    %files ***.sty. The package ``homework'' above was created by me. The other packages
    %are very common for this type of document. You can google to learn more about what
    %they can do, and what options they give you. For example
\usepackage{textcomp}
\usepackage[margin=1.5in]{geometry}
    %the geometry package lets you customize the margins of your document.
    % and the 

\usepackage{forest}
  % For drawing the tree illustrating the sample space  
\usepackage{setspace}
    %package gives us the ability to set the line spacing.
\usepackage{moreenum}   % for greek letters in enum

\newtheorem{theorem}{Theorem}
\theoremstyle{definition} 
\newtheorem{problem}[theorem]{Problem}
    %these set up environments for listing things. The numbering is automatic.

    
\newenvironment{solution}[1][Solution]{\begin{doublespace}\textbf{#1.}\quad }{\ \rule{0.5em}{0.5em}\end{doublespace}}
    %this is the environment for writing solutions. Doble spaced, with an end of proof
    %box at the end
    
\title{Discrete Math\\
CSCI 150, Fall 2020\\
Homework 1}
\author{Guy Matz \\
Hunter College}
    %above is the information that goes in the title. Notice the { and }. 
    %the double slashes \\ mean start a new line.


\begin{document} %this means end the preamble (stuff controling the styles above and
%start the content of the document. We can make adjustments as we go. For example,
%\maketitle
\begin{problem} In how many ways can we seat 10 people on a round table.  Be careful for overcounting; for instance, the following seat assignments for 3 people on a round table are equivalent.?\\
\\
If we keep one person fixed and permute the rest we get $(n-1)!$.  So for 10 people we get $9!$
\end{problem}
\newpage
\begin{problem} In how many ways can we choose one red candy and one blue candy out of 13 red and 17 blue candies?
\\\\
Are the candies distinguishable?
\end{problem}
\newpage

\begin{problem} We have 100 points in the plane.  How many line segments do these points define?
\\\\
\Large
For $n$ points, the first point, combined will all of the others, defines $n-1$ line segments, the second point defines $n-2$ line segments and so on.  Therefore, $n$ points define $\sum_{i=1}^{n-1} = (n-1)* \dfrac{n}{2}$ line segments.  With $n=100$ we have $99*50 = 4950$ points
\end{problem}
\newpage

\begin{problem} A  lottery  ticket  has  the  numbers $\{1,2, . . . ,60\}$ and  one  must  choose  5 numbers to play.  How many possible tickets can we make?
\\\\
\Large
$${60 \choose 5}$$
\end{problem}
\newpage

\begin{problem} Given the digits $\{0,1,2,3,4,5,6,7,8,9\}$, how many 5 digit numbers can we make with different digits?
\\\\
Since the 5 digit number cannot start with 0, we have 9 options for ther first digit, then 9 options for the second digit, 8 options for the third, 7 for the 4th and 6 for the 5th,
$$ 9 * 9 * 8 * 7 * 6 = 27216$$
\end{problem}
\newpage

\begin{problem} How many words of length 7 (not necessarily found in the dictionary) can we make using the alphabet ${a, b, . . . , z}$.
\\\\
\Large
$$26^7$$
\end{problem}
\newpage

\begin{problem} How many subsets of $S=\{1,2,3, . . . , n\}$ don’t have 1?
\\\\
\Large
$$\frac{2^n}{2}$$
\end{problem}
\newpage

\begin{problem} True of False:  The number of ways we can seat 5 people on 10 chairs is more than the number of ways we can seat 10 people on 5 chairs.
\\\\
\Large
False.  They are equal.
\end{problem}
\newpage

\begin{problem} Mr.  and Mrs.  Zetha want to choose a name for their baby such that its mono-gram (first, middle, and last initials) is in alphabetical.  Observe the last letter is always Z.
\\
\\
\begin{enumerate}
\item How  many  such  monograms  are  possible  if  all  letters  must  be  different?  Count in two ways:
\begin{enumerate}
\item using the addition principle by dividing the possibilities into disjoint categories based on the starting letter.
\end{enumerate}
\begin{enumerate}
\item using the product principle by breaking the task of making a name in tw ophases.  Explain your procedure and possible over counting.
\end{enumerate}


\newpage
\item What if the requirement that the monogram must be in alphabetical order is dropped (but letters must still be different)? Use any method to find the answer.


\newpage
\item What  if  letters  can  now  repeat  (but  must  still  be  in  alphabetical  order)?Use any method to find the answer.
\end{enumerate}
\end{problem}


\end{document}
