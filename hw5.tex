\documentclass[10pt,leqno ]{article}
    %The document class defines the master templates, the structure of the document, 
    %and lays out the types of 
    %objects that can be manipulated for this type of document. 
    %the brackets contain basic options that will be applied globally (throughout
    %the document). Here, we specify a 10pt font, and when we number an equation, the 
    %number will be on the left.
    %The document class is a file ***.cls. You will probably never have to edit or create
    % a .cls file. There are many available on the internet for your use.

%%%%%%%%%%%%%%%%%%%%%%%%%%%%%%%%%%%%%%%%%%%%%%%%%%%%%%%%%%%%%%%%%%%%%%%%%%%%%%%%%%%%%%%%%%%%%%%%%%%%%%%%%%%%%%%%%%%%%%%%%%%%%%%%%%%%%%%%%%%%%%%%%%%%%%%%%%%%%%%%%%%%%%%%%%%%%%%%%%%%%%%%%%%%%%%%%%%%%%%%%%%%%%%%%%%%%%%%%%%%%%%%%%%%%%%%%%%%%%%%%%%%%%%%%%%%
\usepackage{amsfonts}
\usepackage[shortlabels]{enumitem}
\usepackage{amssymb}
\usepackage{amsmath}
\usepackage{times}
\usepackage{amsthm}
\usepackage{hyperref}
%\usepackage{homework}
    %packages control the ``style'' or look of the document. These come in the form of 
    %files ***.sty. The package ``homework'' above was created by me. The other packages
    %are very common for this type of document. You can google to learn more about what
    %they can do, and what options they give you. For example
\usepackage{textcomp}
\usepackage[margin=1.5in]{geometry}
    %the geometry package lets you customize the margins of your document.
    % and the 

\usepackage{forest}
  % For drawing the tree illustrating the sample space  
\usepackage{setspace}
    %package gives us the ability to set the line spacing.
\usepackage{moreenum}   % for greek letters in enum
\usepackage{xcolor}

\newtheorem{theorem}{Theorem}
\theoremstyle{definition} 
\newtheorem{problem}[theorem]{Problem}
    %these set up environments for listing things. The numbering is automatic.
\usepackage{float}
\restylefloat{table}

\newenvironment{solution}[1][Solution]{\begin{doublespace}\textbf{#1.}\quad }{\ \rule{0.5em}{0.5em}\end{doublespace}}
    %this is the environment for writing solutions. Doble spaced, with an end of proof
    %box at the end
    
\title{Discrete Math\\
CSCI 150, Fall 2020\\
Homework 5}
\author{Guy Matz \\
Hunter College}
    %above is the information that goes in the title. Notice the { and }. 
    %the double slashes \\ mean start a new line.


\begin{document} %this means end the preamble (stuff controling the styles above and
%start the content of the document. We can make adjustments as we go. For example,
%\maketitle
\begin{problem} An odd number is an integer that can be written as $2k + 1$ where $k \in Z$.  Is -1 an odd number?
\\\\
\Large
With $k=-1$, -1 can be written as $2k + 1$: $2(-1) + 1 = -1$.  So, yes, -1 is an odd number.

\end{problem}
\newpage

\begin{problem} What is the contrapositive of “If it is raining, then I have my umbrella”.
\\
\Large
If it is not raining, then I do not have my umbrella.
\end{problem}
\newpage

\begin{problem} Is the following True or False?
$$ \forall x, y \in  \mathbb{Z}.(x > y) \implies (x^2 > y^2)$$
\\\\
\Large
False!!  Counter-example: x=-1, y=-2
\\\\
To make the statement true, it could read:
$$ \forall x, y \in  \mathbb{N}.(x > y) \implies (x^2 > y^2)$$
\end{problem}
\newpage

\begin{problem} Show that there exist two irrational numbers $x, y$ such that $xy$ is rational.
\\\\
\Large
Let $x = y^{-1}$.  Then $xy = 1$, and 1 is rational
\end{problem}
\newpage

\begin{problem} It is a  fact that if $n^2$ is a multiple of 3, then n is
a multiple of 3. Use this fact to show by contradiction that $\sqrt{3}$  is irrational.
\\\\
\Large

\end{problem}
\newpage

\begin{problem} Prove by contradiction that $17n + 2$ is odd $\implies n$ is odd.
\\\\
\Large

\end{problem}
\newpage


\begin{problem} Show using a direct proof and logical operators the following set equality.
$$A − B^C = A ∩ B$$
\\\\
\Large

\end{problem}
\newpage


\begin{problem} Given a set $S \subset \{1, 3, 5, 7, 9, .\dots \}$ Show that
$$ |P(S)| = X i∈S i ⇒ |S| is even
\\\\
\Large

\end{problem}
\newpage


\begin{problem} 
\\\\
\Large

\end{problem}
\newpage


\begin{problem} 
\\\\
\Large

\end{problem}
\newpage


\begin{problem} 
\\\\
\Large

\end{problem}
\newpage


\end{document}
