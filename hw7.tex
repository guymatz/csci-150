%%%%%%%%%%%%%%%%%%%%%%%%%%%%%%%%%%%%%%%%%%%%%%%%%%%%%%%%%%%%%%%%%%%%%%%%%%%%%%%%%%%%
%%%%%%%%%%%%%%%%%%%free online editor available at%%%%%%%%%%%%%%%%%%%%%%
%%%%%https://www.writelatex.com/%%%%%%%%%%%%%%%%%%%%%%%%%%%%%%%%%%%%%%%%%%

\documentclass[10pt,leqno ]{article}
    %The document class defines the master templates, the structure of the document, 
    %and lays out the types of 
    %objects that can be manipulated for this type of document. 
    %the brackets contain basic options that will be applied globally (throughout
    %the document). Here, we specify a 10pt font, and when we number an equation, the 
    %number will be on the left.
    %The document class is a file ***.cls. You will probably never have to edit or create
    % a .cls file. There are many available on the internet for your use.

%%%%%%%%%%%%%%%%%%%%%%%%%%%%%%%%%%%%%%%%%%%%%%%%%%%%%%%%%%%%%%%%%%%%%%%%%%%%%%%%%%%%%%%%%%%%%%%%%%%%%%%%%%%%%%%%%%%%%%%%%%%%%%%%%%%%%%%%%%%%%%%%%%%%%%%%%%%%%%%%%%%%%%%%%%%%%%%%%%%%%%%%%%%%%%%%%%%%%%%%%%%%%%%%%%%%%%%%%%%%%%%%%%%%%%%%%%%%%%%%%%%%%%%%%%%%
\usepackage{amsfonts}
\usepackage[shortlabels]{enumitem}
\usepackage{amssymb}
\usepackage{amsmath}
\usepackage{times}
\usepackage{amsthm}
\usepackage{hyperref}
%\usepackage{homework}
    %packages control the ``style'' or look of the document. These come in the form of 
    %files ***.sty. The package ``homework'' above was created by me. The other packages
    %are very common for this type of document. You can google to learn more about what
    %they can do, and what options they give you. For example
\usepackage{textcomp}
\usepackage[margin=1.5in]{geometry}
    %the geometry package lets you customize the margins of your document.
    % and the 

\usepackage{forest}
  % For drawing the tree illustrating the sample space  
\usepackage{setspace}
    %package gives us the ability to set the line spacing.
\usepackage{moreenum}   % for greek letters in enum
\usepackage{xcolor}

\newtheorem{theorem}{Theorem}
\theoremstyle{definition} 
\newtheorem{problem}[theorem]{Problem}
    %these set up environments for listing things. The numbering is automatic.
\usepackage{float}
\restylefloat{table}
\usepackage{mathtools}
\DeclarePairedDelimiter{\ceil}{\lceil}{\rceil}
\DeclarePairedDelimiter{\floor}{\lfloor}{\rfloor}

\newenvironment{solution}[1][Solution]{\begin{doublespace}\textbf{#1.}\quad }{\ \rule{0.5em}{0.5em}\end{doublespace}}
    %this is the environment for writing solutions. Doble spaced, with an end of proof
    %box at the end
    
\title{Discrete Math\\
CSCI 150, Fall 2020\\
Homework 1}
\author{Guy Matz \\
Hunter College}
    %above is the information that goes in the title. Notice the { and }. 
    %the double slashes \\ mean start a new line.


\begin{document} %this means end the preamble (stuff controling the styles above and
%start the content of the document. We can make adjustments as we go. For example,
%\maketitle
\begin{problem} Prove that in a group of 25 people, some three of them must have birthdays in the same month.
\\\\
\Large
By the pigeon-hole principle we have $n=12, m = 25$.  Then $\ceil{\frac{25}{12}}= 3$.
\end{problem}
\newpage

\begin{problem} (This  is  pigeonhole  in  reverse)  A  person  needs  to  collect  100  identical coupons.   There  are  6  types  of  coupons  available.   How  many  coupons must the person collect in order to guarantee 100 identical coupons?
\\\\
\Large

$$\ceil{\frac{m}{6}} = 100$$

$$99 < \frac{m}{6} \leq 100$$
$$594 < m \leq 600$$
$$m = 595$$

\end{problem}
\newpage

\begin{problem}  (This is not pigeonhole) My drawer has $n$ pairs of socks, fresh from the laundry but all separated.  Each sock is labeled “Left” or “Right”.  If I grab socks blindly, how many trials do I need to guarantee a good pair?
\\\\
\Large
If $n$ is even you cannot guarantee a "good pair", since you could grab $n$ left socks, then $n$ right socks resulting in no pairs.
\\\\
If $n$ is odd, picking $n+1$ socks will guarantee a pair.
\end{problem}
\newpage

\begin{problem} We  are  given  a  $1 \times l$ rectangle,  where $l$ is  an  integer.   We  place $l+ 1$ points on the perimeter of the rectangle.  Prove that some 2 points must be within a distance of $\sqrt{2}$.  [This second part is not graded, but convince yourself that $l$ points are not sufficient to make the above claim]
\\\\
\Large
Let us think of the $l \times 1$ rectangle as being made up of $l$ $1 \times 1$ squares. Then by the pigeon-hole principle we know that we can place $l$ dots, with each one on the perimeter of their own $1 \times 1$ square.  Again by the pigeon-hole principle, once we place the $l+1$ dot it will have to go into a $1 \times 1$ box that already has a dot on it, and the farthest away those two dots can be is the length of the diagonal of a $1\times 1$ box, which is $\sqrt{2}$.
\end{problem}
\newpage

\begin{problem} Consider a binary word of length 70.  Prove that there are at least two occurrences of a sequence of 6 bits
\\\\
\Large
There are $2^n$ permutations of a 6-digit binary string, which is 64, so there are 64 different ways to write a sequence of 6 bits.  The pigeon-hole principle tells us that once we have 64 digits in a binary string we either have a duplicate 6-digit binary string in those first 64 digits, or the 6-digits have all been unique and the next 6 will be a repeat of one of the 6-digit number in the first 64.
\end{problem}
\newpage

\begin{problem} There is a contest with 40 Pokemons.  There are 18 Pokemons who like to fight in the sky, and 23 show like to fight on ground.  Several of them like to fight in the water.  The number of those who like to fight in the sky and on ground in 9.  There are 7 Pokemons who like to fight in the sky and in water, and 12 who like to fight on ground and in water.  There are 4 Pokemons who like to fight in the sky, on ground, and in water.  How many Pokemons like to fight in water?
\\\\
\Large
Let:\\
$S$ = Sky Fighters\\
$G$ = Ground Fighters\\
$W$ = Water Fighters\\
$SG$ = Sky/Ground Fighters\\
$SW$ = Sky/Water Fighters\\
$GW$ = Ground/Water Fighters\\
$SGW$ = Sky/Ground/Water Fighters\\\\
Then by the Inclusion-Exclusion principle we have:
\begin{align*}
40 =& |S| + |G| + |W| - |SG| - |SW| - |GW| + |SGW|\\
40 =& 18 + 23 + |W| - 9 - 7 - 12 + 4\\
|W| =& 23\\
\end{align*}
\end{problem}
\newpage

\begin{problem}  In a class of 20, 12 are boys and 13 wear glasses.  There are twice as many boys with glasses as girls with no glasses.  How many girls wear glasses?
\\\\
\Large
There are 8 girls, and the number of girls who do not wear glasses is $x$, so the number of girls who wear glasses is $8-x$.  There are twice as many boys with glasses as girls with no glasses - $2x$ - and there are 13 total who wear glasses.  This gives us:
$$13 = (8-x) + 2x$$
Solving for $x$ we find that 5 girls do not wear glasses, so 3 girls do wear glasses.
\end{problem}
\newpage

\begin{problem} Consider making a 3-digit number such that the first digit cannot be 1,and the second digit cannot be 2, and the third digit cannot be 3.  Using the product rule it’s easy to show that we have 8×9×9 = 648 possible numbers.   Show  that  you  can  get  the  same  answer  by  transforming  the problem into an OR logic and using Inclusion-Exclusion.Note:  You answer must have the following format:
$$\square - (\square + \square + \square - \square - \square - \square + \square) = \square$$
where each box is an integer.  We will grade the above format.
\\\\
\Large
$S$ = Total 3-digit numbers\\
$S_1$ = 3-digit numbers with 1 in the hundreds column\\
$S_2$ = 3-digit numbers with 2 in the 10s column\\
$S_3$ = 3-digit numbers with 3 in the 1s column\\
$S_1 \cup S_2$ = 3-digit numbers with 1 in the hundreds column and 2 in the 10s column\\
$S_1 \cup S_3$ = 3-digit numbers with 1 in the hundreds column and 3 in the 1s column\\
$S_2 \cup S_3$ = 3-digit numbers with 2 in the 10s column and 3 in the 1s column\\
$S_1 \cup S_2 \cup S_3$ = 3-digit numbers with 1 in the hundreds column, 2 in the 10s column and 3 in the 1s column\\
\\
Then by the Inclusion-Exclusion principle we have:
$$|S| - (|S_1| + |S_2| + |S_3| - |S_1 \cup S_2| - |S_1 \cup S_3|- |S_3 \cup S_3| + |S_1 \cup S_2 \cup S_3|)$$
$$900 - (100 + 90 + 90 - 10 - 10 - 9 + 1) = 648$$
\end{problem}
\newpage

\begin{problem} Consider the following grid:
\begin{displaymath}
\begin{array}{c c c c}
\bigcirc & \bigcirc & \bigcirc & \bigcirc \\
\bigcirc & \bigcirc & \bigcirc & \bigcirc \\
\bigcirc & \bigcirc & \bigcirc & \bigcirc \\
\bigcirc & \bigcirc & \bigcirc & \bigcirc \\
\end{array}
\end{displaymath}
Experiment with this activity:  Color five of the points black.  For each pair of black points, draw the (infinite) line that joins them.  Observe that some three lines intersect in a right angle triangle.  If you attempt to prove this property (that some three lines intersect in a right angle triangle), what technique will you use and why?  Prove this property.
\\\\
\Large
Pigeon-Hole Principle!  With 4 rows and columns, when we color 5 dots we have necessarily colored in two dots in one column and two dots in one row.  That gives us a horizontal and vertical line which make up the two sides of a right triangle. 
\end{problem}
\newpage

\begin{problem} Redo exercise 8 with one additional constraint:  all digits must be different (use the same Inclusion-Exclusion technique).  Your answer must have the same format described in exercise 8, and this is what will be graded.
\\\\
\Large
$S$ = 3-digit numbers with conditions set in Problem 8\\
$S_D$ = 3-digit numbers with the same digits in hundreds \& tens place (but not 1xx or x2x)\\
$S_E$ = 3-digit numbers with the same digits in hundreds \& ones place (but not 1xx or xx3)\\
$S_F$ = 3-digit numbers with the same digits in tens \& ones place (but not x2x or xx3)\\
\\
Then by the Inclusion-Exclusion principle we have:
$$|S| - (|S_D| + |S_E| + |S_F| - |S_D \cup S_E| - |S_D \cup S_F|- |S_E \cup S_F| + |S_D \cup S_E \cup S_F|)$$
$$648 - (70 + 70 + 63 - 7 - 7 - 6 + 1) = 464$$
\end{problem}
\newpage

\begin{problem} Prove by induction the following property for all integers $n \geq 2$:
$$\sum_{i=2}^{n}0.5^i = 0.5-0.5^n$$
Hint:  Applying the proof by induction steps should be easy, the tricky part is how  to  manipulate  the  expression  for $\sum_{i=2}^{n+1}0.5^i$ in  the  inductive  step  to  get what we want.
\\\\
\Large
Base Case, where $n_0=2$:
$$\sum_{i=2}^{2}0.5^i = 0.5 - 0.5^2 = 0.25 \, \checkmark$$
\\
Inductive Hypothesis:  For some $n \geq n_0$
$$\sum_{i=2}^{n}0.5^i = 0.5-0.5^n$$
Inductive Step:
\begin{align*}
\sum_{i=2}^{n+1}0.5^i =& \sum_{i=2}^{n}0.5^i + 0.5^{n+1} \\
                      =&   0.5-0.5^n + 0.5^{n+1} \\
                      =& 0.5 + 0.5^n(-1 + 0.5)\\
                      =& 0.5 + 0.5^n(-.5)\\
                      =&   0.5 - 0.5^{n+1} \, \checkmark
\end{align*}

\end{problem}
\newpage
\end{document}
