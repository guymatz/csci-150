%%%%%%%%%%%%%%%%%%%%%%%%%%%%%%%%%%%%%%%%%%%%%%%%%%%%%%%%%%%%%%%%%%%%%%%%%%%%%%%%%%%%
%%%%%%%%%%%%%%%%%%%free online editor available at%%%%%%%%%%%%%%%%%%%%%%
%%%%%https://www.writelatex.com/%%%%%%%%%%%%%%%%%%%%%%%%%%%%%%%%%%%%%%%%%%

\documentclass[10pt,leqno ]{article}
    %The document class defines the master templates, the structure of the document, 
    %and lays out the types of 
    %objects that can be manipulated for this type of document. 
    %the brackets contain basic options that will be applied globally (throughout
    %the document). Here, we specify a 10pt font, and when we number an equation, the 
    %number will be on the left.
    %The document class is a file ***.cls. You will probably never have to edit or create
    % a .cls file. There are many available on the internet for your use.

%%%%%%%%%%%%%%%%%%%%%%%%%%%%%%%%%%%%%%%%%%%%%%%%%%%%%%%%%%%%%%%%%%%%%%%%%%%%%%%%%%%%%%%%%%%%%%%%%%%%%%%%%%%%%%%%%%%%%%%%%%%%%%%%%%%%%%%%%%%%%%%%%%%%%%%%%%%%%%%%%%%%%%%%%%%%%%%%%%%%%%%%%%%%%%%%%%%%%%%%%%%%%%%%%%%%%%%%%%%%%%%%%%%%%%%%%%%%%%%%%%%%%%%%%%%%
\usepackage{amsfonts}
\usepackage[shortlabels]{enumitem}
\usepackage{amssymb}
\usepackage{amsmath}
\usepackage{times}
\usepackage{amsthm}
\usepackage{hyperref}
%\usepackage{homework}
    %packages control the ``style'' or look of the document. These come in the form of 
    %files ***.sty. The package ``homework'' above was created by me. The other packages
    %are very common for this type of document. You can google to learn more about what
    %they can do, and what options they give you. For example
\usepackage{textcomp}
\usepackage[margin=1.5in]{geometry}
    %the geometry package lets you customize the margins of your document.
    % and the 

\usepackage{forest}
  % For drawing the tree illustrating the sample space  
\usepackage{setspace}
    %package gives us the ability to set the line spacing.
\usepackage{moreenum}   % for greek letters in enum
\usepackage{xcolor}

\newtheorem{theorem}{Theorem}
\theoremstyle{definition} 
\newtheorem{problem}[theorem]{Problem}
    %these set up environments for listing things. The numbering is automatic.
\usepackage{float}
\restylefloat{table}
\usepackage{mathtools}
\DeclarePairedDelimiter{\ceil}{\lceil}{\rceil}
\DeclarePairedDelimiter{\floor}{\lfloor}{\rfloor}

\newenvironment{solution}[1][Solution]{\begin{doublespace}\textbf{#1.}\quad }{\ \rule{0.5em}{0.5em}\end{doublespace}}
    %this is the environment for writing solutions. Doble spaced, with an end of proof
    %box at the end
    
\title{Discrete Math\\
CSCI 150, Fall 2020\\
Homework 8}
\author{Guy Matz \\
Hunter College}
    %above is the information that goes in the title. Notice the { and }. 
    %the double slashes \\ mean start a new line.


\begin{document} %this means end the preamble (stuff controling the styles above and
%start the content of the document. We can make adjustments as we go. For example,
%\maketitle
\begin{problem} Consider the Fibonacci sequence given by
$$0,1,1,2,3,5,8,13,21, \dots $$
where $F_0 = 0$ and $F_1= 1$ and
$$F_n = F_{n-1} + F_{n-2}$$
We want to show the property that for all $n \geq 0$:\\
$$\sum_{i=0}^{n} F_i = F_0 + F_1 + \dots + F_n = F_{n+2}-1$$
Prove  the  property  by  induction.   Your  proof  must  consists  of  three  parts:  1)  base case(s),  2)  a  clear  statement  of  the  inductive hypothesis,  and  3)  the induction step.  In particular, your inductive step must start with:
$$ \sum_{i=0}^{n+1} F_i = \sum_{i=0}^{n} F_i + F_{n+1} = \dots = \dots = F_{n+3} - 1 = F_{(n+1)+2}-1 $$
Note: Explain why the single base case of $n_0 = 0$ is enough for this proof, even though the recurrence itself requires two base cases to define.\\
Note:  Does the property hold for $n=-1$?  What do you think?
\\\\
\Large
Base Case: $n = 0$,
$$\sum_{i=0}^{0} F_i = F_0 = F_{0+2} -1 = 0$$
Inductive Hypothesis:  For some $n \geq 0$, assume:
$$\sum_{i=0}^{m} F_i = F_{m+2}-1$$
for all $m$, where $0 \leq m \leq n$.\\\\
Inductive Step:
\begin{align*}
\sum_{i=0}^{n+1} F_i &= F_{i} + F_{n+1}\\
                      &= F_{n+2} - 1 + F_{n+1}\\
                      &= F_{n+3} - 1 \\
                      &= F_{(n+1) + 2} - 1
\end{align*}
This recurrence is only defined for $n \geq 0$, so does not hold for $n = -1$.  As for the use of a single base case, this is appropriate since the proof does not make use of the Fibonacci recurrence.
\end{problem}
\newpage

\begin{problem} An  alien  species  communicate  using  a  three-letter  alphabet$\{x,y,z\}$.   In  their language,  words  must  obey  one  single  rule:$zz$ cannot  be  part  of  any  word;otherwise, the speaker will go to sleep and never finish the sentence.  One might wonder how many words of length $n$ exist in this language...
\\\\
Describe the number of words of length $n$ by a recurrence.  Let $a_n$ 
be the number of words of length $n$, and express $a_n$ in terms of $a_{n-1}$ and $a_{n-2}$.
\\\\
Hint:  Do as we did with the tiling problem, i.e.  consider different cases based on how you start a word, then for each case figure out in how many ways you can finish it.
\\\\
\Large
$a_n$ is the number of words with length $n$.\\
Case 1: Word starts with $x$.  Rest of word is of length $n-1$\\
Case 2: Word starts with $y$.  Rest of word is of length $n-1$\\
Case 3: Word starts with $z$.  Then it is followed by a $x$ or $y$, and the rest of word is of length $n-2$\\
$$a_n = a_{n-1} + a_{n-1} + a_{n-2} + a_{n-2} = 2_{a_n - 1} + 2_{a_n - 2}$$
We can see that $a_0 = 1$ (1 way to make a word of no letters), and $a_1 = 3$ (3 ways to make a word of 1 letter).
\end{problem}
\newpage

\begin{problem} Consider a version of the Tower of Hanoi where each disk is duplicated, so we have $2n$ disks with 2 disks of each size.  The rules of the game are the same.  Let $a_n$ be the number of moves needed to solve this $2n$ disk problem.
\\\\
\begin{enumerate}
\item Find a recurrence for $a_n$.
\item Guess a solution for $a_n$ in terms of $n$ (by exploring), and prove it by induction.
\end{enumerate}
\Large
SKIP
\end{problem}
\newpage

\begin{problem} 
Consider the recurrence
$$a_n = 2a_{n-1}+ 2a_{n-2}$$
where $a_0= 1$ and $a_1= 3$.
\\\\
Prove by induction that
$$a_n = \left(\frac{1}{2}+\frac{\sqrt{3}}{3} \right) \rho^n + \left( \frac{1}{2}-\frac{\sqrt{3}}{3} \right) \left(2- \rho \right)^n$$
where $\rho = 1 + \sqrt{3}$.
\\\\
Hint:  Knowing that $\frac{2}{\rho}+\frac{2}{\rho^2}= \frac{2}{2 - \rho}+\frac{2}{(2 - \rho)^2} = 1$ will simplify your inductive step [this is  strikingly similar to the Fibonacci problem in Note 5].
\\
Note:  Explain why the case base must cover $n= 0$ and $n= 1$.
\Large
\\\\
Base Cases,\\
$n = 0$:
$$a_0 = \left( \frac{1}{2} + \frac{\sqrt{3}}{3} \right) + \left( \frac{1}{2} = \frac{\sqrt{3}}{3} \right) = 1$$
$n = 1$:
$$a_1 = \left( \frac{1}{2} + \frac{\sqrt{3}}{3} \right) (1 + \sqrt{3}) + \left( \frac{1}{2} = \frac{\sqrt{3}}{3} \right) (1 - \sqrt{3}) = 3$$
Inductive Hypothesis:\\
For a fixed $n \geq 0$, assume that the recurrence $a_m$ holds for every $m$, where $0 \leq m \leq n$.\\\\
Inductive Step
\begin{align*}
a_{n+1} &= 2a_n + 2a_{n-1} \\
                            &= 2 \left[ 
                                    \left( \frac{1}{2} + \frac{\sqrt{3}}{3}\right)\rho^n
                                    + \left( \frac{1}{2} - \frac{\sqrt{3}}{3}\right)(2-\rho^n)
                                \right]\\
                            & \;\;\;\; + 2 \left[ 
                                \left( \frac{1}{2} + \frac{\sqrt{3}}{3}\right)\rho^{n-1}
                                + \left( \frac{1}{2} - \frac{\sqrt{3}}{3}\right)(2-\rho)^{n-1}
                                \right]\\
                        &= \left( \frac{1}{2} + \frac{\sqrt{3}}{3}\right)\rho^{n+1} \left[  \frac{2}{\rho^2} + \frac{2}{\rho}\right] 
                            + \left( \frac{1}{2} - \frac{\sqrt{3}}{3}\right) (2-\rho)^{n+1} \left[  \frac{2}{(2-\rho)^2} + \frac{2}{2-\rho}\right]\\
                        &= \left( \frac{1}{2} + \frac{\sqrt{3}}{3}\right)\rho^{n+1}
                            + \left( \frac{1}{2} - \frac{\sqrt{3}}{3}\right) (2-\rho)^{n+1} \\
\end{align*}
\end{problem}
\newpage

\begin{problem} We have seen in lecture the expression
$$\frac{2n^2 - 1 +(-1)^n}{8}$$
We want to show the following property is true for all $n \geq 0$:
$$P(n): \frac{2n^2 - 1 + (-1)^n}{8} = \floor{\frac{n}{2}} \ceil{\frac{n}{2}}$$
In order to do prove the claim in the inductive step for $P(n+ 1)$, it is convenient here to use the truth of the claim for $P(n-1)$ (traditionally we would use $P(n))$.  For your inductive step, therefore, start with
$$\frac{2(n+1)^2 - 1 + (-1)^{n+1}}{8} = \frac{2(n-1+2)^2 - 1 + (-1)^{n-1}}{8} = \dots$$
\\\\
Then expand while keeping $(n-1)$ intact in order to apply $P(n-1)$ to obtain:
$$\floor{\frac{n-1}{2}} \ceil{\frac{n-1}{2}}+n$$
and to retrieve $(n+ 1)$ we write:
\Large
SKIP
\end{problem}


\end{document}
